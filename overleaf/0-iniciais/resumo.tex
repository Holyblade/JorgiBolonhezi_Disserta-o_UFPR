\begin{resumo}

O consumo de energia é um dos maiores desafios na infraestrutura de processamento de \emph{Big Data}. Atualmente os gastos com energia são ainda maiores do que na aquisição do \emph{hardware}, representando 75\% do custo total dos \emph{data centers}. Aproximadamente 30\% de toda a energia do \emph{data center} é consumida por \emph{switches} de rede. O \emph{Energy Efficient Ethernet} é um padrão recente que visa reduzir o consumo de energia, embora a prática atual na indústria seja desativá-lo em produção, pois pode causar sobrecargas na rede e perda de desempenho. Esta dissertação fornece uma visão geral de como a atual versão do Apache Hadoop, a 3.x, se comporta com o \emph{Energy Efficient Ethernet} habilitado para links de 1GbE até 400GbE. Os resultados apresentados mostram que há economia de energia significativa com pouca ou nenhuma perda de desempenho para conexões de até 40GbE. No entanto, conexões de 100GbE e 400GbE apresentam perdas significativas de desempenho devido ao despertar do link para transmissões de um único \emph{frame}.



%Cada dia mais a produção de dados pela sociedade tem aumentado, o que implica em maior demanda de recursos e tecnologias viáveis para armazenar, processar e extrair informações úteis de todos estes dados. Uma das alternativas encontradas para tal tarefa foi a criação de \emph{clusters} Hadoop, que utilizam o \emph{framework MapReduce} e podem ser implementados com \emph{hardwares} de baixo custo. Entretanto, um outro problema permanece, o consumo massivo de energia em \emph{data centers}, que tem sido um assunto muito discutido recentemente pela comunidade e indústria. Algumas pesquisas foram realizadas buscando soluções para tal problema, como a criação do EEE-802.3az, um protocolo de rede conhecido como \emph{Green Ethernet}, e o E-EON - \emph{Energy-Efficient and Optimized Networks}, um conjunto de técnicas para redução do consumo de energia no Hadoop, diminuição da latência e aumento da taxa de transferência de rede. Esta pesquisa tem como objetivo principal avaliar e reduzir o consumo de energia da atual versão do Hadoop MapReduce, a 3.x, e para que isso seja alcançado, utilizaremos o NetSLS em testes de diversas configurações de rede sugeridas pelo E-EON como \emph{Ethernet Energy Efficient}, Coalescência de Pacotes e \emph{Active Queue Management}, buscando assim identificar a melhor combinação de rede possível para esta versão.

\end{resumo}