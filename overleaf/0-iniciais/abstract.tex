\begin{abstract}

Energy consumption is one of the major challenges on the big data processing infrastructure. The energy expenses are even higher than hardware, accounting for 75\% of the total cost of nowadays data centers. Narrowing, approximately 30\% of all data center energy is consumed by the network switches. Energy Efficient Ethernet is a recent standard aiming at reduce network power consumption, notwithstanding the current practice in industry is to disable it in production use, since it can cause network overloads and  performance loss. This thesis provides an overview on how Apache Hadoop 3.x, the current version, behaves with Energy Efficient Ethernet enabled for links from 1GbE up to 400GbE links. Presented results show that there is significant energy savings with little or no performance loss for connections up to 40GbE. Nevertheless, connections of 100GbE and 400GbE present significant performance losses due to link wake up to single transmissions.

%Everyday, the data production by society has increased, which results in a greater demand for viable resources and technologies to store, process and extract useful information from all these data. An alternative found for this task are the creation of Hadoop clusters, that uses the MapReduce framework and can be implemented on low cost hardware. However, there's another persistent problem, a massive energy consumption in data centers, which has been a subject much discussed recently by community and industry. Some research was carried out looking for solutions to this problem, such as the EEE-802.3az creation, a network protocol known as Green Ethernet, and Energy-Efficient and Optimized Networks, a techniques set for reducing energy consumption on Hadoop, decreased latency and increased network throughput. The main objective in this research are evaluate and reduce the energy consumption of the current version of Hadoop MapReduce, the 3.x, and for that to be achieved, we will use NetSLS in tests of various network configurations suggested by E-EON as Ethernet Energy Efficient, Packet Coalesing and Active Queue Management for try to identify the best possible network combination for this version. %

\end{abstract}