\chapter{Conclusão}

Um dos principais problemas da infraestrutura de processamento de \emph{big data} é o consumo de energia, que facilmente atinge a cifra de bilhões de kWh consumidos por ano \cite{raizada2020worldwide}. Em 2006, os \emph{data centers} dos EUA consumiram 3 bilhões de kWh \cite{mahadevan2011energy}, enquanto em 2019 os \emph{data centers} globais atingiram 360 bilhões de kWh consumidos \cite{raizada2020worldwide}. As previsões indicam que o consumo de energia pode chegar a 3.000 bilhões de kWh em 2030 \cite{raizada2020worldwide}.

Estudos mostram que os custos com energia tornaram-se maiores do que os com \emph{hardware}, respondendo por 75\% do custo total dos \emph{data centers} \cite{belady2007data}. Em média 10\% a 50\% do consumo de energia do \emph{data center} é por DCNs \cite{abts2010energy}, e esse número tende a ser maior quando recursos como CPU e memória não são totalmente utilizados. Os principais dispositivos consumidores de energia das redes são os \emph{switches}, consumindo cerca de 30\% de toda a energia do \emph{data center} \cite{kliazovich2012greencloud}.

Os \emph{switches} que interligam os links são responsáveis por 65\% do consumo total de energia da rede, e não cessam o consumo mesmo quando não há tráfego na rede \cite{kliazovich2012greencloud}. Conforme relatado em \cite{mahadevan2009energy}, é possível reduzir esse “desperdício” de energia.

O padrão \emph{Energy Efficient Ethernet} (EEE) visa reduzir o consumo de energia das conexões Ethernet configurando um modo de hibernação de link conhecido como \emph{Low Power Idle}. Esta dissertação foi a primeira a testar e analisar o impacto do EEE em links de 1GbE, 10GbE, 25GbE, 40GbE, 100GbE e 400GbE na versão atual do \emph{Hadoop MapReduce}, 3.x. Atualmente o Apache Hadoop \cite{ApacheDocumentation} - uma implementação de código aberto do \emph{MapReduce} \cite{dean2004mapreduce} - é uma das ferramentas de \emph{big data} mais utilizadas para armazenamento e processamento de dados no mundo.

Ao aplicarmos o EEE nos links, encontramos perdas significativas de desempenho para conexões de 100GbE e 400GbE. Estas perdas ocorreram porque em determinados momentos é necessário acordar o link para transmitir um único \emph{frame}, o que causa penalidades de consumo de energia e desempenho em termos relativos que são agravadas de acordo com a largura de banda \cite{jiang2021modeling}; \cite{reviriego2009performance}; \cite{reviriego2010burst}; \cite{e2017energy}. Para contornar tal problema, aplicamos a técnica conhecida como \emph{Packet Coalescing}, um agrupamento de pacotes como forma de limitar o número de interrupções de recebimento, juntamente a dois algoritmos de \emph{Active Queue Management}, o \emph{Controlled Delay} e o \emph{Random Early Detection} para evitar o congestionamento da conexão, e por fim, adicionamos o \emph{Explicit Congestion Notification} para evitar o descarte de pacotes TCP.

Para tanto, avaliamos a execução de \emph{Small Tasks} e \emph{Batch Jobs} com todas essas configurações em diversos \emph{clusters} simulados e encontramos as maiores economias de energia ao utilizarmos \emph{Energy Efficient Ethernet}, com \emph{Packet Coalescing}, \emph{Random Early Detection} e \emph{Explicit Congestion Notification} entre 74\% e 84\% para links de até 400GbE sem perda considerável de desempenho, entre 0,21\% e 1.77\%. Com esses resultados\footnote{Todos os resultados estão disponíveis em \href{https://github.com/holyblade/MestradoUFPR-Jorgi}{https://github.com/holyblade/MestradoUFPR-Jorgi.}}, é possível argumentar que há uma economia de energia significativa com pouca ou nenhuma perda de desempenho para conexões de até 400GbE, o que é interessante se analisarmos em termos de custo-benefício.

\section{Trabalhos Futuros}

Esta dissertação apresenta contribuições para profissionais que realizam pesquisas com \emph{Green Networking} e trabalham na área de \emph{Big Data}. Como trabalhos futuros, pretende-se continuar testando diferentes tamanhos de \emph{cluster}, com outras taxas de assinatura excessiva e topologias, como por exemplo a tradicional hierárquica \emph{fat-tree} com rede de três camadas. Por fim, aplicar todas as configurações recomendadas para balanceamento entre economia de energia e desempenho em outros \emph{frameworks} que seguem o modelo \emph{MapReduce}.

Uma vez que as conexões de 50GbE e 200GbE ainda não foram lançadas pelas Juniper \cite{QFX5220Guide}, também há possibilidades de considerar essas larguras de banda em pesquisas futuras.

\section{Publicação dos Resultados}

Os resultados do Capítulo 4 foram pulicados na \emph{\textbf{2022 IEEE International Conference on Big Data and Smart Computing (BigComp)}}\footnote{\href{http://www.bigcomputing.org/}{http://bigcomputing.org/.}} na modalidade \emph{regular paper} \cite{dias2022reducing}. Para eventos futuros pretendemos submeter um novo artigo apresentando os resultados do Capítulo 5.
